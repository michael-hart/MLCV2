%-------------------------------------------------------------------------------
% Document & Package declarations
%-------------------------------------------------------------------------------

\documentclass[a4paper, 10pt, conference]{ieeeconf}
\usepackage{graphicx}
\usepackage[colorlinks=true, allcolors=black]{hyperref}
\usepackage{tabularx}

%% Language and font encodings
\usepackage[english]{babel}
\usepackage[utf8x]{inputenc}
\usepackage[T1]{fontenc}

%% Useful packages
\usepackage{amsmath}
\usepackage{graphicx}
\usepackage[colorinlistoftodos]{todonotes}
% IEEEConf includes settings for caption/subcaption already!
% \usepackage[font=footnotesize,labelfont=bf]{caption}
\usepackage[font=footnotesize,labelfont=bf]{subcaption}

%% Packages for displaying source code
\usepackage{listings}
% \usepackage[framed,numbered,autolinebreaks,useliterate]{mcode}

\usepackage{float}
\usepackage{longtable}

%% Packages for displaying source code
\usepackage[numbered,framed]{matlab-prettifier}
\usepackage{color}

%%*************************************************************************
%% Legal Notice:
%% This code is offered as-is without any warranty either expressed or
%% implied; without even the implied warranty of MERCHANTABILITY or
%% FITNESS FOR A PARTICULAR PURPOSE!
%% User assumes all risk.
%% In no event shall IEEE or any contributor to this code be liable for
%% any damages or losses, including, but not limited to, incidental,
%% consequential, or any other damages, resulting from the use or misuse
%% of any information contained here.
%%
%% All comments are the opinions of their respective authors and are not
%% necessarily endorsed by the IEEE.
%%
%% This work is distributed under the LaTeX Project Public License (LPPL)
%% ( http://www.latex-project.org/ ) version 1.3, and may be freely used,
%% distributed and modified. A copy of the LPPL, version 1.3, is included
%% in the base LaTeX documentation of all distributions of LaTeX released
%% 2003/12/01 or later.
%% Retain all contribution notices and credits.
%% ** Modified files should be clearly indicated as such, including  **
%% ** renaming them and changing author support contact information. **
%%
%% File list of work: IEEEtran.cls, IEEEtran_HOWTO.pdf, bare_adv.tex,
%%                    bare_conf.tex, bare_jrnl.tex, bare_jrnl_compsoc.tex,
%%                    bare_jrnl_transmag.tex
%%*************************************************************************

%-------------------------------------------------------------------------------
% Document Configuration
%-------------------------------------------------------------------------------

\begin{document}
\title{Machine Learning for Computer Vision - Image Matching}
\author{Michael~Hart (00818445) and
        Meng~Kiang~Seah (00699092)
\\
        Department of Electrical and Electronic Engineering,
        Imperial College London,
        SW7 2AZ
\\
        E-mail: \{mh1613, mks211\}@imperial.ac.uk}
\date{\today}

%-------------------------------------------------------------------------------
% Plan on what to write
%-------------------------------------------------------------------------------

% See coursework instructions at:
% https://bb.imperial.ac.uk/bbcswebdav/pid-1034737-dt-content-rid-3589968_1/courses/DSS-EE4_62-16_17/MLCVCoursework2.pdf

%-------------------------------------------------------------------------------
% Information Banner
%-------------------------------------------------------------------------------

\maketitle

%-------------------------------------------------------------------------------
% Abstract
%-------------------------------------------------------------------------------

\begin{abstract}
Lorem ipsum dolor sit amet, consectetur adipiscing elit. Phasellus gravida viverra sollicitudin. Nulla ornare enim in ante auctor rhoncus a vel nulla. Nulla condimentum massa rhoncus, sodales arcu a, euismod nulla. Proin viverra mauris at massa molestie, a ultricies tortor fermentum. Duis consectetur, ante a tincidunt euismod, augue diam varius dolor, ut vestibulum orci est sit amet mi. Aenean sit amet metus vitae sem malesuada tempus. Vivamus placerat ornare erat quis tincidunt. In quis massa aliquet, pellentesque magna vitae, luctus eros.

\end{abstract}

%-------------------------------------------------------------------------------
% Introduction
%-------------------------------------------------------------------------------
\section{Introduction}
% Paper considers what?
% What data is used?
% What are the methods discussed? Short explanations

\section{Question 1 - Matching}
\subsection{Manual}
% Mike: Quick discussion? Or just point them to the code in the appendix.

Although automatic interest point selection is preferable for computer vision, a valuable tool for checking how the algorithms work is to allow a user to manually select interest points. Five pairs of points are selected by the user, which are then assumed to match, removing the need for automatic point selection and patch matching. The MATLAB code to do this is found in the Appendix as file \texttt{q1\_manual.m}.


\subsection{Automatic}
\begin{enumerate}
    \item Harris Detector Method. PIC: img1 with points on it, show with different thresholds.
    \item Patch Extraction and Features Method. PIC: Example patches extracted, resulting histograms example.
    \item NN Matching Method PIC: Using img1 and img2 with
    \texttt{showMatchedFeatures(imgA, imgB, matchedPoints1, matchedPoints2);} to show result.
\end{enumerate}

\subsection{Transformation Estimation}
\begin{enumerate}
    \item Homography Matrix Estimation Method PIC: img1 with img2
    \item Fundamental Matrix Estimation Method
    \item ErrorHA Method (also does projecting)
    \item Needs to be done lol. ``Implement a method for calculating epipolar line given point coordinates and a
Fundamental matrix between two images. Display that line on an image in Matlab.''
\end{enumerate}

\section{Question - Image Geometry}
\subsection{Homography with HG Pictures}
\subsubsection{Reduced Size}
0.5995 Error
\subsubsection{Manual vs. Automatic}
Harris 4K points for 500 threshold
\subsubsection{Number of Correspondences}

\subsection{Stereo Vision With FD Pictures}


\cite{notes}


%-------------------------------------------------------------------------------
% Conclusion
%-------------------------------------------------------------------------------
\section{Conclusion}
Lorem ipsum dolor sit amet, consectetur adipiscing elit. Phasellus gravida viverra sollicitudin. Nulla ornare enim in ante auctor rhoncus a vel nulla. Nulla condimentum massa rhoncus, sodales arcu a, euismod nulla. Proin viverra mauris at massa molestie, a ultricies tortor fermentum. Duis consectetur, ante a tincidunt euismod, augue diam varius dolor, ut vestibulum orci est sit amet mi. Aenean sit amet metus vitae sem malesuada tempus. Vivamus placerat ornare erat quis tincidunt. In quis massa aliquet, pellentesque magna vitae, luctus eros.

%-------------------------------------------------------------------------------
% References
%-------------------------------------------------------------------------------
\bibliographystyle{unsrt}
\bibliography{mlcv_refs}

%-------------------------------------------------------------------------------
% Appendix(ces)
%-------------------------------------------------------------------------------
\onecolumn
\section*{Appendix}

\subsection*{q1\_manual.m}
\lstinputlisting[style=Matlab-editor]{src/q1_manual.m}
\newpage

\end{document}
